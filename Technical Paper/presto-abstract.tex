\documentclass[sigconf]{acmart}

%\usepackage{booktabs} % For formal tables


% Copyright
\setcopyright{none}
%\setcopyright{acmcopyright}
%\setcopyright{acmlicensed}
%\setcopyright{rightsretained}
%\setcopyright{usgov}
%\setcopyright{usgovmixed}
%\setcopyright{cagov}
%\setcopyright{cagovmixed}

\settopmatter{printacmref=false} % Removes citation information below abstract
\renewcommand\footnotetextcopyrightpermission[1]{} % removes footnote with conference information in first column
\pagestyle{plain} % removes running headers



\begin{document}
\title{Presto}

\author{Ankita Alshi}
\affiliation{%
  \institution{Indiana University Bloomington}
  \city{Bloomington}
  \state{Indiana}
  \postcode{47404}
}
\email{aralshi@iu.edu}


% The default list of authors is too long for headers.



\begin{abstract}
Presto is a SQL query engine developed specially for interactive analytics. It
focuses on large commercial data warehouses with capacity of gigabytes to
petabytes. It is open source and used for distributed systems. It is compatible
with relational as well as NoSQL data sources such as Cassandra and 
Hive~\cite{hid-sp18-502-Presto}.

It is being used by big organizations like Facebook to run interactive queries
against their large data warehouses. The main advantage of using Presto is that
it allows to perform analytics on data from different data sources using single
query. This allows data to be combined across organizations without extra
overhead of separate queries for each data source~\cite{hid-sp18-502-Presto}.
\end{abstract}

%
% The code below should be generated by the tool at
% http://dl.acm.org/ccs.cfm
% Please copy and paste the code instead of the example below.
%



\keywords{SQL, Hadoop, Map-reduce, Big data, HDFS}


\maketitle

\section{Introduction}
All the big organizations usually have large data warehouses and different data sources which needs to be well integrated to provide efficient use of them. Now a days there is a high demand of interactive and quick query processing for analytical processes. It is become necessary to use SQL on Hadoop so that larger group of organizations can use the Hadoop on commodity clusters to fulfill their technology need. Presto is one such open source SQL engine that provides the functionality to interact with distributed database with very low latency.\\
Hive can also be used for running SQL queries on big data systems but it takes longer time to execute the queries. Fast query execution of presto is achieved by complete pipelining of query execution. Also Presto is a in memory SQL query execution in engine which means that data on which query is executed resides in main memory of the nodes. This eliminates the time required to access that data from the disk. disadvantage of in-memory query execution is that it is not fault tolerant as the intermediate query results are not stored on any non-volatile storage device. Also there is a restriction on the size of the query as not all the data can be stored in memory of the nodes. So large queries which require data more than the size of the main memory would not be executed successfully.\\
Presto is implemented in Java as it is easy to integrate it with other systems and maintain the code as well.

\section{Architecture}
Presto is designed to provide speed for analytical queries. Architecture of presto is very simple. Client queries are sent to the coordinator server. It first parses the SQL query using Metadata APIs. The parsed query is goes through the planning stage where its execution is planned such that it can be executed in pipeline. After that coordinator uses data location APIs to schedule the tasks to run on different worker nodes such that each worker node is execute query on its local data stored in main memory. \\
Once tasks are schedule all the worker nodes execute each task in pipelined mode. The query is divided into stages and data is passed on from one stage to another as it is ready. This ensures that at a time multiple queries are executed by each node providing additional level of parallelism to achieve faster results. Pipelining is done across the network such that any available node can execute different stages. The data transfer is from main memory to main memory so it takes less time. when the result are ready worker node sends it back to the client. Coordinator server keeps monitoring the query execution. 


\begin{figure*}
\includegraphics[height=8cm, width=15cm,keepaspectratio]{presto-architecture}
\caption{Presto Architecture}
\end{figure*}


\section{Concepts}
\subsection{Server Types}
There are two types of servers in presto architecture- Coordinator and Worker. Both the types of servers are different set of responsibilities that they work towards. 
\subsubsection{Coordinator}
Presto coordinator server can be considered by main backbone of the system. Coordinator server responsible for managing the worker server present in presto setting. All the client queries are routed to coordinator server. Coordinator server contains all the information of the worker servers, where they are located and the data that they locally store. This information is used wisely by the server to plan the query into task and to schedule these tasks across the nodes on the network such that data required for executing these task is either present locally or very close to the worker server. Additionally Coordinator server also keeps track of query execution once the worker nodes starts processing it.\\
Coordinator server parse the SQL query and creates a logical model out of it such that it can be executed in series to stages. These stages are nothing but the task which are executed on distributed worker nodes available in the cluster. Each Presto installation needs at least one Coordinator server.

\subsubsection{Worker}
Worker server is mainly responsible for processing the queries assigned to it by the coordinator. Worker node fetches the data required for the query processing from connectors or other worker nodes. When a new presto worker node starts up, it informs the coordinator server in the presto installation. This way the coordinator server gets to know that new worker is added to the installation and can be considered to schedule tasks to run on it.\\
Worker node executes the query as it is planned by the coordinator. Intermediate results generated by worker nodes are passed on to other worker node on which the next execution stage is scheduled. Final results generated by worker node is sent back to the coordinator server which sends it back to the client.

\subsection{Data Sources}
Presto id designed to work with different types of data sources. Connectors are used in order to run SQL query against all the types data sources. Connectors are very similar to database drivers, they provide set of APIs to connect to that database. Presto contains connectors to connect to Hive, HBase, relational databases and NoSQL data sources as well. Presto keeps catalogs to resolve the queries. Catalogs are used to store the schema for different connectors. Catalogs are used to reference the connectors. Schema is nothing but way of organizing the data in different data sources. schema for a relational database is table whereas for a NoSQL database is a document. Catalogs and schema are used to defined traditional set of tables against which the SQL queries can be processed.

\subsection{Query Execution}
Client sends SQL statements to the coordinator. This SQL statement can not be executed as it is by the worker. Coordinator server parses this statement into distributed query plan. Distributed query plan can be visualized as a tree with leaf nodes processing small tasks and parent of those nodes executing aggregated functions on their results. At the end root node aggregates results of all nodes in the tree to create the final output.\\
Distributed query plan is created as a set of interconnected stages. Data flows from one stage to another. Each stage might involve in retrieving data from different data sources using connectors. At the end of each stage an intermediate output is generated. Instead of running a complete stage on one node,the stage is further divided into tasks and these tasks are scheduled to run on worker nodes. 

\section{Use Case}

\section{Conclusions}

%\end{document}  








\bibliographystyle{ACM-Reference-Format}
\bibliography{sample-bibliography}

\end{document}

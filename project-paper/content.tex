% status: 0
% chapter: TBD

\title{REST Service for Azure object management}


\author{Ankita Alshi}
\affiliation{%
  \institution{Indiana University}
  \city{Bloomington} 
  \state{IN} 
  \postcode{47404}
  \country{USA}}
\email{aralshi@iu.edu}


% The default list of authors is too long for headers}
\renewcommand{\shortauthors}{Ankita Alshi}

\begin{abstract}
Microsoft uses its data center to host a cloud platform called Azure to
provide platform as a service or infrastructure as a service. It is very
difficult and costly for small companies to manage their own data center.
Microsoft Azure helps such companies to focus on their services and takes care
of the infrastructure for them. Azure can be used to design, develop and deploy
variety of applications~\cite{hid-sp18-502-microsoft-azure}. The NIST Big Data
 Reference Architecture (NBDRA) provides an interface that contains definitions
 of important resources related to cloud computing in Json
 format~\cite{hid-sp18-502-nist-vol8}. Our goal is to create rest service that
 will help users identify the properties of Azure virtual machines and other
 resources available on Azure platform.

\end{abstract}

\keywords{hid-sp18-502, Swagger, Virtual Machine, Azure}


\maketitle

\section{Introduction}
The NIST Big Data Reference Architecture defines objects necessary to
understand deployment on Microsoft Azure server. A REST service can be defined
to add, delete and update instances of these objects. Microsoft Azure
provides different services such as storage, compute, networking,
containers, Databases, Analytics~\cite{hid-sp18-502-microsoft-azure}. Along with
 the objects related to Azure given in NIST Big Data Reference Architecture, we
 aim to identify more objects such that different services provided by Azure can
 be covered. Object definitions will be done using swagger specification in
 Json format. This specification will be used to generate REST service server
 side and client side code using swagger codegen. Each object cab be accessed
 by the REST service using its own base path. Base path for the objects will be
 \emph{cloudmesh/azure} followed by object name.


\section{Object Definitions}
\begin{itemize}

\item Azure User Credentials: Username and password for authentication to allow
access to objects.

\item Azure Virtual Directories: Azure can be used for storage purpose as well.
This object will contain virtual directories created by a user.

\item Azure Files: The files stored on Azure under the virtual directories.

\item Azure size: Size of the Azure machine image in terms of resources like
number of cores, disks, primary and secondary memory, bandwidth and price.

\item Azure image: image od different operating systems that can be installed on
 virtual machines

\item virtual machine: Key properties of Azure virtual machines including
username, its status, operating system image running on it.


\end{itemize}



\bibliographystyle{ACM-Reference-Format}
\bibliography{report} 

